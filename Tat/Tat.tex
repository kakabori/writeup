\documentclass[12pt,letterpaper]{article}

%%%%%%%%%%%%%%%%% common packages are included here %%%%%%%%%%%%%%%%%
\usepackage{graphicx}
\usepackage{epstopdf}
\usepackage{tabularx}
\usepackage[lofdepth,lotdepth]{subfig} % subfloat
\usepackage{bm} % bold math
\usepackage{color}
\usepackage[centertags]{amsmath}
\usepackage{mathrsfs}
\usepackage{amsmath}
\usepackage{mathtools}
\usepackage{amssymb}
\usepackage{amsfonts}
\usepackage{amssymb}
\usepackage{amsthm}
\usepackage{newlfont}
\usepackage{textcomp,gensymb} % for \textcelsius, \textdegree, and \degree
\usepackage{syntonly}

%%%%%%%%%%%%%%%%%%%% some declaration %%%%%%%%%%%%%%%%%%%%%
\DeclareGraphicsExtensions{.pdf,.png,.jpg,.eps}
\graphicspath{{./figures/}}

%%%%%%%%%%%%%%%%%%% new commands are defined here %%%%%%%%%%%%%%%%%%%%%
\newcommand{\dg}{$^{\circ}$} % degree symbol
\newcommand{\iang}{\AA$^{-1}$} % inverse Angstrom symbol
\newcommand{\degC}{$^{\circ}\mathrm{C}$} % degree Celcius
\newcommand{\Eq}[1]{Eq.\,(\ref{#1})} % reference to an equation

% Some mathematical (physical) quantities and symbols that are used often
\newcommand{\xhat}{\mathbf{\hat{x}}}
\newcommand{\yhat}{\mathbf{\hat{y}}}
\newcommand{\zhat}{\mathbf{\hat{z}}}
\newcommand{\kin}{\mathbf{k}_{\mathrm{in}}}
\newcommand{\kout}{\mathbf{k}_{\mathrm{out}}}
\newcommand{\Tat}{\mathrm{Tat}}
\newcommand{\DOPC}{\mathrm{DOPC}}
\newcommand{\cm}{\mathrm{cm}}

% To simplify some formatting issues
\newcommand{\pars}[1]{\mathopen{}\left( #1 \right)\mathclose{}} % () without extra spaces due to \left and \right
\newcommand{\angles}[1]{\left\lange #1 \right\rangle}%      <>
\newcommand{\braces}[1]{\left\lbrace #1 \right\rbrace}%     {}
\newcommand{\bracks}[1]{\left\lbrack #1 \right\rbrack}%     []
\newcommand{\ds}[1]{\displaystyle{#1}}%
\newcommand{\+}{^{\dagger}}%                                
\newcommand{\partiald}[3][]{{\partial^{#1}#2 \over \partial {#3}^{#1}}}%


%\setlength{\oddsidemargin}{0cm} \setlength{\topmargin}{0cm}
%\setlength{\textheight}{22cm} \setlength{\textwidth}{16cm}
%-----------------------------------------------------------------

\begin{document}
\today

\section{MD simulation}
Three systems with different petpide concentrations (DOPC:Tat = 128:0, 128:2, and 128.4)
were studied with Gromacs 4.6.1 package \cite{ref:Hess08}. DOPCs were modeled by Slipid 
force field \cite{ref:Jambeck12_JPCB,ref:Jambeck12_JCTC} and HIV-Tats
were modeled by Amber99SB \cite{ref:Hornak06}. The systems were simulated at 
310 K with a constant
area in the x-y plane. The z direction was coupled to 1 atm with constant pressure. The
center of mass (COM) distance between each peptide and the bilayer was constrained by an
umbrella potential with a force constant of 3000 kJ/mol/nm$^2$. Each system was explored
18 independent simulations as a combination of 3 different constant area and 6 different
peptide insertion depths (except the pure DOPC system).

\begin{equation}
  z_{\mathrm{cm}} = \frac{\sum_{i=1}^N m_iz_i}{\sum_{i=1}^N m_i}
\end{equation}
The center of mass constraint was applied through an external force field,
which derives from an added pontential energy of the system. The pontential 
is like a spring, where it depends on the deviation of the distance 
between the center of mass of Tat and DOPC from a prefered value, $z_0$,
\begin{equation}
  U(z_1^{\Tat},\ldots,z_1^{\DOPC},\ldots) = 
  -\frac{1}{2} k 
  \pars{z_{\cm}^{\Tat} - z_{\cm}^{\DOPC} - z_0}^2
\end{equation}
Then, $-\partial U/\partial z_i$ is equal to the external force acting 
on atom, $i$. Before applying this constraint, Tats were attached to 
the bilayer from the water region. During the first 20 ns for 
pre-equilibration, Tats were allowed to change their configuration,
which resulted in different configurations for each Tat when attached
to the bilayer. It is possible that the Tats configuration inside the
bilayer at the end of simulations was affected by this initial 
configuration of Tats. Instead of preparing many simulations with
different initial Tat configurations, we avaraged over all Tats
present in the system. We also performed many simulations with
different $A_L$ and $z_{\Tat}$ to investigate how robust some of the 
Tat structural features are. Many simulation results are shown in
the appendix of this thesis. 

\section{Cylinder Model}
Tat is modeled as a cylinder. Lipids are seprated into three regions, 
suppressed, boundary, and unperturbed lipids (see Fig. for the definitions
of various quantities). Let $h(r)$ represent the phosphorus height profile,
which is given by
\begin{equation}
  h(r) = \left\{ 
  \begin{array}{cc}
    Z_{Phos} & \text{if } 0 < r < R_1 \\
    \frac{Z_{Phos}-Z_{Phos}^0}{R_1-R_2}r+b & \text{if } 0 < r < R_1 \\
    Z_{Phos}^0 & \text{if } 0 < r < R_1 
  \end{array}\right.  
\end{equation}     
To obtain $R_3$, we assume that the simulation box is cylindrical, so that
$NA_L=\pi R_3^2$, or $R_3=\sqrt{NA_L/\pi}$. $R_1=\sqrt{V_{Tat}/(\pi h_{Tat}}$.

\section{Some Lipid Bilayer Data}
We also estimated the structure by fitting the experimental form factors to a 
model using the SDP method with the component groups identified in Fig. (what  
The positions of 
these groups were free parameters and the agreement with the experimental form 
factors was excellent (see Fig. S.M. 5).  Absolute total electron density 
profiles and the Tat profiles are shown for many samples in Fig. 6 (A-C).  
It must be emphasized, however, that, while the total EDP is well determined by 
this fitting procedure.  Indeed, there are local minima in the fitting landscape, 
including one with Tat closer to the center of the bilayer as shown in Fig. 
S.M. 5.  The simulations help to discard that result. For the results shown in 
Fig. 6, a consistent trend is that Tat moves away from the bilayer center as 
concentration increases. 

Figure shows that $A_L$ as defined by $(V_L-V_{HL})/D_c$ decreases as the 
amount of DOPE increases for systems without Tat. This is consistent with the
previous studies (or predictions?) and attributed to the small size of PE
head group. Because DOPE has smaller head group than DOPC, lipids in DOPC/DOPE
bilayers pack more compactly than DOPC bilayers do, leading to a smaller $A_L$.
Consequently, bilayers composed of DOPC and DOPE tend to have a higher order 
parameter than DOPC alone. (NO THEY DON'T. WHAT'S GOING ON HERE?). Similarly,
the thickness of bilayers is larger at higher PE content. 

Figure shows that Tat is located further out from the bilayer center with 
higher content of PE lipids. This is also consistent with MD simulation PMF,
which showed that arginine insertion cost more energy for PE membrane than
PC membrane, the result of which was attributed to more possible hydrogen 
bonding between PE group and arginines.



More structural detail from the modeling and from the simulations is shown in 
Fig. 7. The bilayer thickness can be described as DHH, which is the 
distance between the maxima in the electron density profile, or as DPP, which 
is the distance between the phosphocholines on the opposing monolayers. Figs. 
7A and 7B show that both these quantities decrease with increasing Tat mole 
fraction (P/(L+P)), showing that Tat thins membranes, increasingly so as its 
concentration is increased, even though both simulation and modeling suggest 
that Tat moves further from the membrane center with increasing concentration 
as shown in Fig. 7D.  Fig. 7C shows that the area per lipid AL usually increases 
with increasing mole fraction of Tat, as would be expected from consideration of 
conservation of lipid volume. Interestingly, the bilayer thickness did not 
increase for DOPC/DOPE (3:1) bilayers with x less than 0.03.  

Figure 8 shows that the Sxray orientational order parameters generally 
decreases with increasing concentration of Tat for most of the membrane mimics 
studied.  These decreases in membrane chain order are compatible with the 
increase in softening of membranes by Tat observed by a decrease in the bending
energy, KC, in Fig. 2.

\bibliography{../thesis}
\bibliographystyle{unsrt}

\end{document}

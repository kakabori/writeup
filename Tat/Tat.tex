\documentclass[12pt,letterpaper]{article}
\author{Kiyo Akabori}
\title{Mosaic Spread Analysis}
\date{\today}
%-------------------------------------Package-----------------------
\usepackage{graphicx}
\usepackage{epstopdf}
\usepackage{tabularx}
\usepackage[lofdepth,lotdepth]{subfig}%subfloat
\usepackage{bm}%bold math
\usepackage{color}
\usepackage[centertags]{amsmath}
\usepackage{mathrsfs}
\usepackage{amsmath,amssymb}
\usepackage{amsfonts}
\usepackage{amssymb}
\usepackage{amsthm}
\usepackage{newlfont}
\usepackage{textcomp,gensymb}%for \textdegree and \degree

\DeclareGraphicsExtensions{.pdf,.png,.jpg,.eps}
\graphicspath{{./figures/}}

%------------------------------new command------------------------             
%\newcommand{\dg}{^{\circ}}%degree symbol
\newcommand{\iang}{\textrm{\AA}^{-1}}%inverse Angstrom symbol
\newcommand{\Eq}[1]{Eq.\,(\ref{#1})}%reference to an equation
\newcommand{\xhat}{\mathbf{\hat{x}}}
\newcommand{\yhat}{\mathbf{\hat{y}}}
\newcommand{\zhat}{\mathbf{\hat{z}}}
\newcommand{\kin}{\mathbf{k}_{\mathrm{in}}}
\newcommand{\kout}{\mathbf{k}_{\mathrm{out}}}

%\setlength{\oddsidemargin}{0cm} \setlength{\topmargin}{0cm}
%\setlength{\textheight}{22cm} \setlength{\textwidth}{16cm}
%-----------------------------------------------------------------

\begin{document}
\maketitle
%\tableofcontents
%\listoffigures
%\listoftables

%\newpage
%\section{Notation}
%\begin{itemize}
%\item $\omega$: angle of incidence
%\item $\hat{\theta}$: angle measured from equator in q-space
%\item $S$, S-distance: distance between the center of the sample and the CCD screen
%\item $D$, D-spacing: repeat spacing of lipid bilayers
%\item $d$, d-spacing: distance between lipid chains
%\item $\theta_t$: chain tilt angle
%\item $\gamma$: $\gamma$ angle of ripple phase [refer to Figure \ref{fig:Sun1996}]
%\item $\xi$: the angle between the midplane of the major side and the ripple direction
%\item X-axis: horizontal axis through the beam in the CCD frame
%\item Z-axis: vertical axis through the beam in the CCD frame
%\item (X,Z): position on the CCD detector with respect to the beam
%\end{itemize}


\section{Some Lipid Bilayer Data}
We also estimated the structure by fitting the experimental form factors to a 
model using the SDP method with the component groups identified in Fig. (what  
The positions of 
these groups were free parameters and the agreement with the experimental form 
factors was excellent (see Fig. S.M. 5).  Absolute total electron density 
profiles and the Tat profiles are shown for many samples in Fig. 6 (A-C).  
It must be emphasized, however, that, while the total EDP is well determined by 
this fitting procedure, the values of the parameters for the components are not 
as well determined as they would be if one had X-ray data to smaller and larger 
q and neutron data.  Indeed, there are local minima in the fitting landscape, 
including one with Tat closer to the center of the bilayer as shown in Fig. 
S.M. 5.  The simulations help to discard that result. For the results shown in 
Fig. 6, a consistent trend is that Tat moves away from the bilayer center as 
concentration increases. 

Figure shows that $A_L$ as defined by $(V_L-V_{HL})/D_c$ decreases as the 
amount of DOPE increases for systems without Tat. This is consistent with the
previous studies (or predictions?) and attributed to the small size of PE
head group. Because DOPE has smaller head group than DOPC, lipids in DOPC/DOPE
bilayers pack more compactly than DOPC bilayers do, leading to a smaller $A_L$.
Consequently, bilayers composed of DOPC and DOPE tend to have a higher order 
parameter than DOPC alone. (NO THEY DON'T. WHAT'S GOING ON HERE?). Similarly,
the thickness of bilayers is larger at higher PE content. 

Figure shows that Tat is located further out from the bilayer center with 
higher content of PE lipids. This is also consistent with MD simulation PMF,
which showed that arginine insertion cost more energy for PE membrane than
PC membrane, the result of which was attributed to more possible hydrogen 
bonding between PE group and arginines.



More structural detail from the modeling and from the simulations is shown in 
Fig. 7. The bilayer thickness can be described as DHH, which is the 
distance between the maxima in the electron density profile, or as DPP, which 
is the distance between the phosphocholines on the opposing monolayers. Figs. 
7A and 7B show that both these quantities decrease with increasing Tat mole 
fraction (P/(L+P)), showing that Tat thins membranes, increasingly so as its 
concentration is increased, even though both simulation and modeling suggest 
that Tat moves further from the membrane center with increasing concentration 
as shown in Fig. 7D.  Fig. 7C shows that the area per lipid AL usually increases 
with increasing mole fraction of Tat, as would be expected from consideration of 
conservation of lipid volume. Interestingly, the bilayer thickness did not 
increase for DOPC/DOPE (3:1) bilayers with x less than 0.03.  

Figure 8 shows that the Sxray orientational order parameters generally 
decreases with increasing concentration of Tat for most of the membrane mimics 
studied.  These decreases in membrane chain order are compatible with the 
increase in softening of membranes by Tat observed by a decrease in the bending
energy, KC, in Fig. 2.  
\end{document}

\documentclass[12pt,letterpaper]{article}
\author{Kiyo Akabori}
\title{Mosaic Spread Analysis}
\date{\today}
%-------------------------------------Package-----------------------
\usepackage{graphicx}
\usepackage{epstopdf}
\usepackage{tabularx}
\usepackage[lofdepth,lotdepth]{subfig}%subfloat
\usepackage{bm}%bold math
\usepackage{color}
\usepackage[centertags]{amsmath}
\usepackage{mathrsfs}
\usepackage{amsmath,amssymb}
\usepackage{amsfonts}
\usepackage{amssymb}
\usepackage{amsthm}
\usepackage{newlfont}
\usepackage{textcomp,gensymb}%for \textdegree and \degree

\DeclareGraphicsExtensions{.pdf,.png,.jpg,.eps}
\graphicspath{{./figures/}}

%------------------------------new command------------------------             
%\newcommand{\dg}{^{\circ}}%degree symbol
\newcommand{\iang}{\textrm{\AA}^{-1}}%inverse Angstrom symbol
\newcommand{\Eq}[1]{Eq.\,(\ref{#1})}%reference to an equation
\newcommand{\xhat}{\mathbf{\hat{x}}}
\newcommand{\yhat}{\mathbf{\hat{y}}}
\newcommand{\zhat}{\mathbf{\hat{z}}}
\newcommand{\kin}{\mathbf{k}_{\mathrm{in}}}
\newcommand{\kout}{\mathbf{k}_{\mathrm{out}}}

%\setlength{\oddsidemargin}{0cm} \setlength{\topmargin}{0cm}
%\setlength{\textheight}{22cm} \setlength{\textwidth}{16cm}
%-----------------------------------------------------------------

\begin{document}
\maketitle
%\tableofcontents
%\listoffigures
%\listoftables

%\newpage
%\section{Notation}
%\begin{itemize}
%\item $\omega$: angle of incidence
%\item $\hat{\theta}$: angle measured from equator in q-space
%\item $S$, S-distance: distance between the center of the sample and the CCD screen
%\item $D$, D-spacing: repeat spacing of lipid bilayers
%\item $d$, d-spacing: distance between lipid chains
%\item $\theta_t$: chain tilt angle
%\item $\gamma$: $\gamma$ angle of ripple phase [refer to Figure \ref{fig:Sun1996}]
%\item $\xi$: the angle between the midplane of the major side and the ripple direction
%\item X-axis: horizontal axis through the beam in the CCD frame
%\item Z-axis: vertical axis through the beam in the CCD frame
%\item (X,Z): position on the CCD detector with respect to the beam
%\end{itemize}


\section{Mosaic Spread: Calculation}
In this section, an analytical framework for a measurement of mosaic spread will 
be developed. Let us imagine that a sample is made up of many small domains 
that are 
tilted from the direction perpendicular to the substrate normal by some amount. 
A "perfect" domain is a domain that is parallel to the substrate plane.
Then, we can consider a probability distribution function, $P(\alpha)$, 
representing a probablity of finding a domain with tilt $\alpha$, which is the 
angle
between the substrate normal and the tilted domain normal. Here, we have
assumed the rotational symmetry about the substrate normal, so that the 
distribution
does not depend on the azimuthal angle, $\beta$. The normalization condition on 
the probability distribution is 
\begin{equation}
  1 = \int_0^{2\pi} \!\! \mathrm{d}\beta  
      \int_0^{\frac{\pi}{2}} \! \mathrm{d}\alpha \, \sin\alpha \, P(\alpha).
\end{equation}
The object of this section is to derive the x-ray scattering structure factor 
including the probability distribtion. The cooridate system employed here is 
such that x, y, and
z-axes of the zero tilt domain, that is, a domain parallel to the substrate, 
coincide with the lab x, y, and z-axes 

First, we want to calculate the structure factor for a domain tilted by 
$\alpha$ 
and $\beta$, expressed in the lab coordinates. See Fig. XXX. For this, we need 
to express
$\mathbf{q}$ in terms of . We imagine
rotating the coorinates about the y-axis first, and then about the z-axis. In
other words, we apply the appropriate rotation matrices to , y, and 
z-axes. The rotation matrix for rotaing a vector about y-axis is given by
\begin{equation}
  \begin{pmatrix} 
    \cos\alpha & 0 & -\sin\alpha \\ 
    0 & 1 & 0 \\
    \sin\alpha & 0 & \cos\alpha 
  \end{pmatrix}
\end{equation}
and for ratating about z-axis
\begin{equation}
  \begin{pmatrix} 
    \cos\beta & \sin\beta & 0 \\ 
    -\sin\beta & \cos\beta & 0 \\
    0 & 0 & 1 
  \end{pmatrix}
\end{equation}
Then, what we want is
\begin{equation}
  \mathbf{\hat{x}}' = 
  \begin{pmatrix} 
    \cos\beta & \sin\beta & 0 \\ 
    -\sin\beta & \cos\beta & 0 \\
    0 & 0 & 1 
  \end{pmatrix}
  \begin{pmatrix} 
    \cos\alpha & 0 & -\sin\alpha \\ 
    0 & 1 & 0 \\
    \sin\alpha & 0 & \cos\alpha 
  \end{pmatrix}
  \begin{pmatrix}
    1 \\
    0 \\
    0
  \end{pmatrix}
  = 
  \begin{pmatrix}
    \cos\alpha\cos\beta \\
    \cos\alpha\sin\beta \\
    -\sin\alpha
  \end{pmatrix}
\end{equation}
\begin{equation}
  \mathbf{\hat{y}}' = 
  \begin{pmatrix} 
    \cos\beta & \sin\beta & 0 \\ 
    -\sin\beta & \cos\beta & 0 \\
    0 & 0 & 1 
  \end{pmatrix}
  \begin{pmatrix} 
    \cos\alpha & 0 & -\sin\alpha \\ 
    0 & 1 & 0 \\
    \sin\alpha & 0 & \cos\alpha 
  \end{pmatrix}
  \begin{pmatrix}
    0 \\
    1 \\
    0
  \end{pmatrix}
  =
  \begin{pmatrix}
    -\sin\beta \\
    \cos\beta \\
    0
  \end{pmatrix}
\end{equation}
\begin{equation}
  \mathbf{\hat{z}}' = 
  \begin{pmatrix} 
    \cos\beta & \sin\beta & 0 \\ 
    -\sin\beta & \cos\beta & 0 \\
    0 & 0 & 1 
  \end{pmatrix}
  \begin{pmatrix} 
    \cos\alpha & 0 & -\sin\alpha \\ 
    0 & 1 & 0 \\
    \sin\alpha & 0 & \cos\alpha 
  \end{pmatrix}
  \begin{pmatrix}
    0 \\
    0 \\
    1
  \end{pmatrix}
  =
  \begin{pmatrix}
    \sin\alpha\cos\beta \\
    \sin\alpha\sin\beta \\
    \cos\alpha
  \end{pmatrix}
\end{equation}
Then, the components of $\mathbf{q}$ represented in the rotated coodinates, 
denoted 
by $\mathbf{q'}$, are the projection of $\mathbf{q}$ on x$'$, y$'$, and 
z$'$-axes, 
that is,
\begin{equation}
  q_x' = \mathbf{q} \cdot \mathbf{\hat{x}'} 
       = q_x\cos\alpha\cos\beta + q_y\cos\alpha\sin\beta -q_z\sin\alpha  
\end{equation}
\begin{equation}
  q_y' = \mathbf{q} \cdot \mathbf{\hat{y}'} 
       = -q_x\sin\beta + q_y\cos\beta  
\end{equation}
\begin{equation}
  q_z' = \mathbf{q} \cdot \mathbf{\hat{z}'} 
       = q_x\sin\alpha\cos\beta + q_y\sin\alpha\sin\beta + q_z\cos\alpha  
\end{equation}
The transformation rule we are looking for is 
\begin{equation}
  \cos\theta' = \frac{q_z'}{q} 
              = \sin\theta\sin\alpha\cos(\phi-\beta) + \cos\theta\cos\alpha 
  \label{eq:theta'}
\end{equation}
and
\begin{equation}
  \tan\phi' 
    = \frac{q_y'}{q_x'}
    = \frac{\sin\theta\sin(\phi-\beta)}{\sin\theta\cos\alpha\cos(\phi-\beta) 
                                       -\cos\theta\sin\alpha}
  \label{eq:phi'}
\end{equation}
The structure factor of the tilted domain in the lab coordinates is simply 
given 
by $S(\mathbf{q'})=S(q,\theta',\phi')$. Summing over all the domains, we get 
for the total structure factor
\begin{equation}
  S_M(q,\theta,\phi) 
    = \int_0^{2\pi} \!\! \mathrm{d}\beta \int_0^{\frac{\pi}{2}} 
      \mathrm{d}\alpha \, \sin\alpha \, S(q,\theta',\phi') \, P(\alpha)
  \label{eq:SM}
\end{equation}
with Eq.\,(\ref{eq:theta'}) and Eq.\,(\ref{eq:phi'}).

For $\theta=0$, $\theta'=\alpha$ and $\phi'=0$ or $\pi$, so we have
\begin{align}
  S_M(q,0) 
    &\sim \int_0^{\frac{\pi}{2}} \mathrm{d}\alpha \, \sin\alpha \, 
          S(q,\alpha,0) \, P(\alpha)
          + \int_0^{\frac{\pi}{2}} \mathrm{d}\alpha \, \sin\alpha \, 
            S(q,\alpha,\pi) \, P(\alpha) \\
    &= \int_{-\frac{\pi}{2}}^{\frac{\pi}{2}} \mathrm{d}\alpha \, \sin\alpha \, 
       S(q,\alpha) \, P(\alpha)
\end{align} 
if we understand $S(q,\alpha)$ to be the structure factor on the ($q_r$,$q_z$) 
plane. This shows that $S_M(q,0)$ is equal to the convolution of the 
distribution
function and the original structure factor. In general, however, $S_M(q,\theta)$ 
is 
not a convolution of the distribution function and the structure factor.

Given Eqs.\,(\ref{eq:theta'}), (\ref{eq:phi'}), and (\ref{eq:SM}), we want to
 show that
mosaic spread acts as one dimensional convolution in the x-ray structure factor:
\begin{equation}
  S_M(q,\theta) 
    = \int_{-\pi}^{\pi} \mathrm{d}\alpha \, S(q,\theta-\alpha) \, P(\alpha)
  \label{eq:conv}
\end{equation}
The structure factor representing Bragg peaks in the spherical coordinates are 
written as
\begin{equation}
  S(q,\theta,\phi) 
    \sim \frac{\delta(q-\frac{2\pi h}{D})}{q^2} 
         \delta(\cos\theta-1) \delta(\phi)
  \label{eq:Bragg}
\end{equation}
where $\delta(x)$ is the Dirac delta function. Plugging \Eq{eq:Bragg} in 
\Eq{eq:SM},
we obtain $\phi-\beta=0$. Using this condition, we get 
\begin{equation}
  S_M(q,\theta) \sim \frac{\delta(q-\frac{2\pi h}{D})}{q^2}P(\theta)\sin\theta,
\end{equation}
which shows that we can directly measure the probability distribution 
experimentally
by looking at the intensity along $q=2\pi h/D$. In the next section, we will 
discuss 
the relevant experimental techniques.

\section{Mosaic Spread: Experiment}
In this section, we discuss experimental procedures to probe appropriate 
$q$-space
to measure the mosaic distribution, $P(\alpha)$. In our setup, the angle of 
incidence between the beam and substrate, denoted by $\omega$, can be varied. A 
conventional method to measure mosaicity distribution is a rocking scan, where
one measures the integrated intensity of a given Bragg peak as a function of 
$\omega$ with a fixed detector position. In a non-conventional method called
ring analysis, one measures the intensity as a function of $\eta$ on a two
dimensional detector. First, we want to show that the two methods mentioned 
above in fact measure the mosaicity disitribution and therefore are equivalent
to each other.

Let $\omega$ be the angle of incidence, $2\theta$ be the total scattering angle,
$p_x$ be the pixel number in the horizontal direction, $p_z$ be the pixel 
number in the vertical direction, $eta$ be the angle measured from the 
$p_z$-axis 
on the detector. $\Delta p$ is 0.07113 mm/pixel. $q_x=\mathbf{q} \cdot 
\hat{\mathbf{x}}$, $q_y=\mathbf{q} \cdot \hat{\mathbf{y}}$, and $q_z=\mathbf{q} 
\cdot \hat{\mathbf{z}}$, 
where $\hat{\mathbf{x}}$, $\hat{\mathbf{y}}$, and $\hat{\mathbf{z}}$ are all 
defined on the sample space. This means that $\hat{\mathbf{y}}$ and 
$\hat{\mathbf{z}}$ both rotate as $\omega$ is varied while $\hat{\mathbf{x}}$ 
is always perpendicular to the beam. What we need is a set of transformation
rules for going from the detector space (pixels) to the sample $q$-space.
With them, we would know how to trace out a line on the detector in order
to measure the mosaicity distribution. 


The incoming and outgoing wavevectors of the x-ray beam in Fig. XXX 
are given by
\begin{equation}
  \kin = \frac{2\pi}{\lambda} \yhat, \quad
  \kout = 
    \frac{2\pi}{\lambda} \left( 
      \sin 2\theta \cos\phi \, \xhat
      + \cos 2\theta \, \yhat
      + \sin 2\theta \sin\phi \, \zhat 
    \right),
  \label{eq:kinkout}
\end{equation}
where $\lambda$ is the wavelength of x-ray. The scattering vector is
the difference between $\kin$ and $\kout$,
\begin{align}
  \mathbf{q} &= \kout - \kin \nonumber \\
             &= q \left( 
                  \cos\theta\cos\phi \, \xhat - \sin\theta \, \yhat
                  + \cos\theta\sin\phi \, \zhat
                \right),
  \label{eq:q_vector}
\end{align}
where $q=4\pi\sin\theta/\lambda$ is the magnitude of the scattering vector. 
When the sample is rotated by $\omega$ about the x-axis in the clockwise 
direction as shown in Fig. XXX, the sample coodinates written in terms of 
the lab coordinates are  
\begin{equation}
  \mathbf{\hat{e}_x} = \xhat, \quad
  \mathbf{\hat{e}_y} = \cos\omega\,\yhat + \sin\omega\,\zhat, \quad
  \mathbf{\hat{e}_z} = -\sin\omega\,\yhat + \cos\omega\,\zhat.
  \label{eq:smp_coord}
\end{equation}
From \Eq{eq:q_vector} and \Eq{eq:smp_coord}, we find the projection of 
$\mathbf{q}$ on the sample coordinates to be
\begin{align}
  q_x &= \mathbf{q}\cdot\mathbf{\hat{e}_x} 
       = q\cos\theta\cos\phi 
       \label{eq:qx} \\
  q_y &= \mathbf{q}\cdot\mathbf{\hat{e}_y} 
       = q\left(-\sin\theta\cos\omega + \cos\theta\sin\phi\sin\omega\right) 
       \label{eq:qy} \\
  q_z &= \mathbf{q}\cdot\mathbf{\hat{e}_z} 
       = q\left(\sin\theta\sin\omega + \cos\theta\sin\phi\cos\omega\right).
       \label{eq:qz}
\end{align}
With respect to the beam, the position on the detector is given by
\begin{equation}
  X = S \tan 2\theta \cos\phi, \quad Z = S \tan 2\theta \sin\phi.
\end{equation} 
The pixels on the detector are directly proportional to $X$ and $Z$. Thus,
these equations define the transformation rules from the detector space
to the sample q-space and vice versa.

In terms of these coordinates, in the rocking scan, $\phi=\pi/2$ and 
$q=2\pi h/D$
while $\omega$ is varied about $\theta_B$, where $\theta_B$ is the Bragg
angle for a Bragg peak that is focused on. Using $q=4\pi\sin\theta/\lambda$, 
we recover the Bragg condition, $2D\sin\theta=h\lambda$. Plugging $\phi=\pi/2$
in Eq.\,(\ref{eq:qx}), (\ref{eq:qy}), and (\ref{eq:qz}), and taking 
$\theta=\theta_B$, 

\end{document}
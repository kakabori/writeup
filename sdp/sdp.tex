\documentclass[letterpaper,12pt]{article}
\author{Kiyo Akabori}
\title{DMPC Ripple Phase}
%-------------------------------------Package-----------------------
\usepackage{graphicx}
\usepackage{epstopdf}

\usepackage{tabularx}
\usepackage[lofdepth,lotdepth]{subfig}%subfloat
\usepackage{bm}%bold math
\usepackage{color}
\usepackage[centertags]{amsmath}
\usepackage{mathrsfs}
\usepackage{amsmath,amssymb}
\usepackage{amsfonts}
\usepackage{amssymb}
\usepackage{amsthm}
\usepackage{newlfont}
\usepackage[export]{adjustbox}

\DeclareGraphicsExtensions{.pdf,.png,.jpg,.eps}
\graphicspath{{./figures/}}

%------------------------------new command------------------------             
\newcommand{\dg}{$^{\circ}$}%degree symbol
\newcommand{\iang}{\AA$^{-1}$}%inverse Angstrom symbol

%\setlength{\oddsidemargin}{0cm} \setlength{\topmargin}{0cm}
%\setlength{\textheight}{22cm} \setlength{\textwidth}{16cm}
%-----------------------------------------------------------------

\begin{document}
%\title{Ripple Phase}
%\author{Kiyo Akabori}
%\today

%\tableofcontents
%\listoffigures
%\listoftables

%\newpage
%\section{Notation}
%\begin{itemize}
%\item $\omega$: angle of incidence
%\item $\hat{\theta}$: angle measured from equator in q-space
%\item $S$, S-distance: distance between the center of the sample and the CCD screen
%\item $D$, D-spacing: repeat spacing of lipid bilayers
%\item $d$, d-spacing: distance between lipid chains
%\item $\theta_t$: chain tilt angle
%\item $\gamma$: $\gamma$ angle of ripple phase [refer to Figure \ref{fig:Sun1996}]
%\item $\xi$: the angle between the midplane of the major side and the ripple direction
%\item X-axis: horizontal axis through the beam in the CCD frame
%\item Z-axis: vertical axis through the beam in the CCD frame
%\item (X,Z): position on the CCD detector with respect to the beam
%\end{itemize}


\section{Absorption Correction}
This section derives a formula to correct the absorption effect due to the sample.
The idea follows Yufeng's thesis, chapter 6. 

Within each domain, there is no x-ray
absorption due to lipids or water. Let the domain size be $L_z$ and thickness $t$.
$t=NL_z$, where $N$ is the total number of domains. Let $I_{meas}=NI_0$ represent
the measured intensity at the detector. Here, $I_0$ is equal to the x-ray intensity
due to each domain. Assume that x-rays travel to the middle of each domain and 
scatter. The path length of x-ray within the top most domain is given by 
$L_z/\sin\theta$, $3L_z/\sin\theta$ within the second domain, and 
$(2i+1)L_z/\sin\theta$ within the $i$th domain. Then, the measured intensity is
given by  
\begin{align}
  I_{meas} 
    &= \sum_{i=0}^{N-1} I_0 \exp
       \left\{-\frac{(2i+1)L_z}{x_a\sin\theta}\right\} \nonumber \\
    &= I_0 \sum_{i=0}^{N-1} ar^i \nonumber \\
    &= I_0 a \frac{1-r^N}{1-r},
\end{align}
where $a=\exp\{-L_z/(x_a\sin\theta)\}$ and $r=a^2$. 
Since $I_{meas}=AI_{ideal}=ANI_0$, the absorption correction, $A_c$ is given by
\begin{align}
  A_c = \frac{1}{A} &= \frac{N}{a} \frac{1-r}{1-r^N}\\
                    &= \frac{t/L_z}{\exp(-\frac{L_z}{x_a\sin\theta})} 
                       \frac{1-\exp(-\frac{2L_z}{x_a\sin\theta})} 
                            {1-\exp(-\frac{2t}{x_a\sin\theta})}.
\end{align}

In the NFIT program, a convenient quantity called scaling factor denoted by $\phi$ 
is defined via  $I_{meas}=\phi(q_z)S(\mathbf{q})$. Then, $\phi(q_z)=A|F(q_z)|^2/q_z$. 
$\phi(q_z)$ is calculated by NFIT and input to the SDP program. The SDP program then
converts $\phi(q_z)$ to $|F(q_z)|$. These two quantities are related by
\begin{equation}
  |F(q_z)|=\sqrt{q_zA_c\phi(q_z)}
\end{equation}
In the SDP program, $L_z$ is assumed to be 2000 \AA. $\sin\theta$ is related to $q_z$
by
\begin{equation}
  \sin\theta = \frac{q_z\lambda}{4\pi},
\end{equation}
where $\lambda$ is the wavelength. This relation assumes that scattering of the 
interest is at $q_r=0$, which is not a bad approximation for LAXS analysis. Also, note
that the derived formula for absorption correction works well for specular scattering,
in which the incident angle is the same as the outgoing angle. A more refined theory
that takes diffuse scattering into account will be considered in the next section. 


\section{Absorption Correction 2}
In this section, a slightly more accurate theory will be derived for the absorption
effect. It will involve an explicit integration over the incident angle, $\omega$,
which is necessiated by the sample rotation during an x-ray exposure. The procedure
is to write down an absorption factor, $A_{\omega}$, for a given incident angle, and
then integrate over $\omega$. For each $\omega$, $A_{\omega}$ will have explicit
dependence on $q_z$ because x-rays hitting different CCD pixels travel different 
paths through the sample. Note that we ignore $q_x$ dependence.

Assume that all the x-rays enter the sample from the top surface. The total scattering
angle is given by $2\theta$. See Figure. Let $z$-axis point downward. At the top surface
(air-sample interface), $z=0$. For x-rays that travel to $z$ and then scatter, the
total path length within the sample is given by
\begin{equation}
  L_{tot} = \frac{z}{\sin\omega}+\frac{z}{\sin(2\theta-\omega)} = zg(\omega).
\end{equation}
For each ray, the intensity is attenuated by the sample. If non-attenuated 
intensity is equal to $I_0$, then the attenuated intensity is
\begin{equation}
  I = I_0\exp\left(-\frac{L_{tot}}{\mu}\right)
\end{equation}
Then, the measured total intensity is given by
\begin{align}
  I_{meas}(\omega) 
    &= \int_0^t I(z) dz
     = I_0\int_0^t \exp\left(-\frac{g(\omega)}{\mu}z\right)dz \nonumber \\
    &= I_0\mu \frac{1-\exp\left(-\frac{t}{\mu}g(\omega)\right)}{g(\omega)}.
\end{align}
Since, for a given $\omega$, $I_{meas}=A_{\omega}I_{ideal}=A_{\omega}tI_0$, 
\begin{equation}
  A_{\omega} = \frac{\mu}{t} 
               \frac{1-\exp\left(-\frac{t}{\mu}g(\omega)\right)}{g(\omega)}.
\end{equation}
Note that $q_z$ dependence is through $\theta$ in $g(\omega)$:
\begin{equation}
  \theta = \arcsin\left(\frac{q_z\lambda}{4\pi}\right).
\end{equation}
The measured intensity including the sample rotation is given by
\begin{equation}
  I_{total}(\theta) = \int_0^{2\theta}d\omega I_{meas}(\omega,\theta),
\end{equation}
so that the total absorption factor is equal to
\begin{equation}
  A(\theta) = \frac{\mu}{2\theta t} \int_0^{2\theta}d\omega 
      \frac{1-\exp\left(-\frac{t}{\mu}g(\omega)\right)}{g(\omega)}.
\end{equation}
If $\mu$ is taken to infinity (no absorption), $A$ goes to 1 as expected. 
Here, it is important to note that $1/2\theta$ factor in the above equation
is normally called Lorentz polarization factor, which
is usually approximated as $1/q_z$ for LAXS analysis. Since the SDP
program applies this correction factor in addition to the absorption
correction, we remove this factor in the formula for $A_c$. Therefore,
the final result for the total absorption correction is 
\begin{equation}
  A_c(\theta) 
    = \frac{1}{2\theta A(\theta)} \nonumber \\
    = \frac{t}{\mu} 
       \left[ 
         \int_0^{2\theta}d\omega 
         \frac{1-\exp\left(-\frac{t}{\mu}g(\omega)\right)}{g(\omega)}
       \right]^{-1}
\end{equation}
with $g(\omega)=1/\sin\omega+1/\sin(2\theta-\omega)$.


\end{document}
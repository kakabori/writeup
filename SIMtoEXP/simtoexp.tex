\documentclass[letterpaper,12pt]{article}
\input{../common_preambles}

%\setlength{\oddsidemargin}{0cm} \setlength{\topmargin}{0cm}
%\setlength{\textheight}{22cm} \setlength{\textwidth}{16cm}

\begin{document}
\today

\section{SIMtoEXP software}
The details of the program can be found in \cite{ref:Kucerka10}. Here, a brief
introduction to the software will be given.

SIMtoEXP is a program to analyze the bilayer structure along the bilayer normal.
It reads in the number density of each atom in lipids and peptides and calculate
the form factor from the bilayer electron density profile,
\begin{equation}
  F(q_z) = \int_{-z_0}^{z_0} \, \rho(z) dz.
\end{equation}
An assumption made here is that x-ray form factor is properly represented by 
the Fourier transform of an average electron density profile, while, strictly,
x-ray form factor is equal to.
Within Born approximation, the X-ray intensity is a statistical (time) average 
over absolute form factor squared,
\begin{equation}
  I(q_z) = \left< |F(q_z)|^2 \right>
\end{equation}

An electron density profile of a DOPC bilayer simulated by S-lipid force field 
is shown in Fig. ref{fig:DOPC}. The exact parsing of atoms into each component
group is shown in Fig. . Parsing with more components
can be achieved, but one shown is the same parsing employed in SDP modelling for
x-ray data, which will be discussed in a later chapter. 

Calculated form factor is shown in . In order to obtain 
reliable structure from simulations, simulations with various area per lipid 
were done. A simulation that best matches the experimental form factor was 
chosen and analyzed to obtain the bilayer thickness. For systems with Tat, 
Tat was fixed at various z coordinates at various area per lipid. Table ref{tb:Tat}
shows chi squared values for all the simulations performed for this study. As 
seen in Table ref{tb:Tat}, smaller chi squared was obtained for larger z values.
While the smallest $\chi^2$ was obtained at $z = 20 \AA$, this simulation was
considered artifactual because it is difficult for the bilayer thickness to be
modified so much for a rather small insertion of Tat. Since the experimental
data already suggest that bilayer thickenss was decreased as increasing 
concentration of Tat, Tat must be embbed by an appreciable amount in the bilayer.

While physical reasoning can be employed to eliminate some of small $\chi^2$ 
simulations, simulations with $A_L=$ and $z_{Tat}=18, 16, 14$ seem to be all good.
As Fig. shows, simulated form factors look very similar
albeit small differece in $\chi^2$. Instead of taking the smallest $\chi^2$ as
the best structure for this system, we averaged three simulations weighted by
their $\chi^2$ as ... 

\bibliography{../thesis}
\bibliographystyle{unsrt}

\end{document}

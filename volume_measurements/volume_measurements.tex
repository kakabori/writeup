\documentclass[letterpaper,12pt]{article}
\author{Kiyo Akabori}
\title{Summary of Volume Measurements}
%-------------------------------------Package-----------------------
\usepackage{graphicx}
\usepackage{epstopdf}

\usepackage{tabularx}
\usepackage[lofdepth,lotdepth]{subfig}%subfloat
\usepackage{bm}%bold math
\usepackage{color}
\usepackage[centertags]{amsmath}
\usepackage{mathrsfs}
\usepackage{amsmath,amssymb}
\usepackage{amsfonts}
\usepackage{amssymb}
\usepackage{amsthm}
\usepackage{newlfont}

\DeclareGraphicsExtensions{.pdf,.png,.jpg,.eps}
\graphicspath{{./figures/}}

%------------------------------new command------------------------             
\newcommand{\dg}{$^{\circ}$}%degree symbol
\newcommand{\iang}{\AA$^{-1}$}%inverse Angstrom symbol

%\setlength{\oddsidemargin}{0cm} \setlength{\topmargin}{0cm}
%\setlength{\textheight}{22cm} \setlength{\textwidth}{16cm}
%-----------------------------------------------------------------

\begin{document}
%\title{Ripple Phase}
%\author{Kiyo Akabori}
%\today

%\tableofcontents
%\listoffigures
%\listoftables

%\newpage
%\section{Notation}
%\begin{itemize}
%\item $\omega$: angle of incidence
%\item $\hat{\theta}$: angle measured from equator in q-space
%\item $S$, S-distance: distance between the center of the sample and the CCD screen
%\item $D$, D-spacing: repeat spacing of lipid bilayers
%\item $d$, d-spacing: distance between lipid chains
%\item $\theta_t$: chain tilt angle
%\item $\gamma$: $\gamma$ angle of ripple phase [refer to Figure \ref{fig:Sun1996}]
%\item $\xi$: the angle between the midplane of the major side and the ripple direction
%\item X-axis: horizontal axis through the beam in the CCD frame
%\item Z-axis: vertical axis through the beam in the CCD frame
%\item (X,Z): position on the CCD detector with respect to the beam
%\end{itemize}

\section{Sample Preparation}
\subsection{Molecular Weight of Tat}
The Tat peptide sequence used in x-ray experiements and MD simulations is 
YGRKKRRQRRR, where one letter notation of amino acids is used. The molecular 
weight of this sequence is 
$181.2+75.1+146.1+2 \times 146.2+6\times 174.2-10\times 18=1560$.
Peptides are normally synthetized in trifluoroacetic acid, which has 
the chemical formula $\mathrm{CF_3CO_2H}$, and made into a powder form by the 
freeze-dry method (ref?). Therefore, each positively charged amino acid such as 
an arginine and lysine is counter-balanced by a trifluoroacetate (TFA)
($\mathrm{C_2F_3O_2}$). Since Tat has six arginines and two lysines, it comes 
with eight trifluoroacetates. This complex has molecular weight of 
$1560+113\times 8=2464$. When Tat peptides are weighed, one must use the 
molecular weight of the complex in order to calculate the molarity of a Tat
in water solution correctly.

\begin{table}[ht]
  \centering
  \begin{tabular}{c c c}
    Molecule & Molecular Weight & Volume \\
    Tat (YGRKKRRQRRR) & 1560 & 1876 \\ 
    Tat + TFA & 2464 & 2964
  \end{tabular}
  \caption{Important Quantities for Tat Peptide}
  \label{tb:Tat}
\end{table}

\begin{table}[ht]
  \centering
  \begin{tabular}{c c c c}
    Code & Amino acid & Chemical Formula & Molecular weight (g/mol) \\
    K & Lysine & $\mathrm{C_6H_{14}N_2O_2}$ & 146.2 \\
    R & Arginine & $\mathrm{C_6H_{14}N_4O_2}$ & 174.2 \\
    G & Glycine & $\mathrm{C_2H_5NO_2}$ & 75.1\\
    Y & Tyrosine & $\mathrm{C_9H_{11}NO_3}$ & 181.2 \\
    Q & Glutamine & $\mathrm{C_5H_{10}N_2O_3}$ & 146.1  
  \end{tabular}
  \caption{Amino Acids Data}
  \label{tb:aa}
\end{table}

\subsection{Volume Measurement}
The volume of Tat was measured using a densimeter, which measure the average 
density of a solution. First, the mass of Tat and water were measured to be 
3.7 and 1212.6 mg via a digital balance. The density of water and 
Tat-water solution were measured to be 0.993325 and 0.99418 $\mathrm{g/cm^3}$,
respectively.

The Tat volume is calculated in the following way. Assuming that Tat
molecules in water does not change the volume of water molecules, the density
of Tat-water solution is equal to the mass of Tat-water solution divided
by the sum of volumes of water and Tat, 
\begin{equation}
  \rho_{sol} = \frac{m_w+m_c}{V_w+V_cN_c},
\end{equation}
where $\rho_{sol}$ is the density of the solution, $m_w$ and $m_c$
are the total mass of water and Tat-TFA complex, $V_w$ is the total volume of 
water, $V_c$ is the volume of a Tat-TFA complex, and $N_c$ is the total number 
of the complex in the solution. Using $V_w=m_w/\rho_w$ and $N_c=N_Am_c/W_c$, after some
simple algebra, we arrive at
\begin{equation}
  V_c = \frac{W_c}{\rho_{sol}N_A} \left( 
        1 + \frac{m_w}{m_c}\left(1-\frac{\rho_{sol}}{\rho_w}\right) 
        \right),
\end{equation}
where $W_c$ is the molecular weight of the complex, $N_A$ is the Avogadro's number,
and $\rho_w$ is the density of water. The measured values of these quantities
are shown in Table \ref{tb:values}. As described in the previous section, 
Tat powder comes with couterions, so the volume measured here is that 
of a Tat-TFA complex. Assuming here for simplicity that the molecular
volume scales with the molecular weight, the volume of Tat was measured to be
2964 $\mathrm{\AA^3} \times$ 1560/2464=1876 $\mathrm{\AA^3}$. This value is in quite a good agreement with the 
value from a peptide calculator website (ref?). As will be shown in a later 
chapter, MD simulations also predict a similar value.

\begin{table}[ht]
  \centering
  \begin{tabular}{c c}
    $\rho_{sol}$ & 0.994180 $\mathrm{g/cm^3}$\\
    $\rho_w$ & 0.993325 $\mathrm{g/cm^3}$\\
    $m_w$ & 1212.6 mg \\
    $m_T$ & 3.73 mg \\ 
  \end{tabular}
  \caption{Measured Quantities in }
  \label{tb:values}
\end{table}
\end{document}
\documentclass[12pt,letterpaper]{article}
\input{../common_preambles}

% Z's
\newcommand{\zw}{Z_\mathrm{W}}
\newcommand{\zchtwo}{Z_\mathrm{CH_2}}
\newcommand{\zh}{Z_\mathrm{H}}
% sigma's
\newcommand{\sigmah}{\sigma_\mathrm{H}}
\newcommand{\sigmam}{\sigma_\mathrm{M}}
% rho's
\newcommand{\rhoh}{\rho_\mathrm{H}}
\newcommand{\rhom}{\rho_\mathrm{M}}
\newcommand{\rhog}{\rho_\mathrm{G}}
\newcommand{\rhos}{\rho_\mathrm{s}}
\newcommand{\rhm}{R_\mathrm{HM}}
\newcommand{\rhow}{\rho_\mathrm{W}}
\newcommand{\rhochtwo}{\rho_\mathrm{CH_2}}
% subscript
\newcommand{\chtwo}{\mathrm{CH_2}}
\newcommand{\w}{\mathrm{W}}

%\syntaxonly
\usepackage[pdftex]{hyperref}


\begin{document}
\today

\section{setup}
The incoming and outgoing wavevectors of the x-ray beam in Fig. XXX 
are given by
\begin{equation}
  \kin = \frac{2\pi}{\lambda} \yhat, \quad
  \kout = 
    \frac{2\pi}{\lambda} \left( 
      \sin 2\theta \cos\phi \, \xhat
      + \cos 2\theta \, \yhat
      + \sin 2\theta \sin\phi \, \zhat 
    \right),
  \label{eq:kinkout}
\end{equation}
where $\lambda$ is the wavelength of x-ray. The scattering vector is
the difference between $\kin$ and $\kout$,
\begin{align}
  \mathbf{q} &= \kout - \kin \nonumber \\
             &= q \left( 
                  \cos\theta\cos\phi \, \xhat - \sin\theta \, \yhat
                  + \cos\theta\sin\phi \, \zhat
                \right),
  \label{eq:q_vector}
\end{align}
where $q=4\pi\sin\theta/\lambda$ is the magnitude of the scattering vector. 
When the sample is rotated by $\omega$ about the x-axis in the clockwise 
direction as shown in Fig. XXX, the sample coodinates written in terms of 
the lab coordinates are  
\begin{equation}
  \mathbf{\hat{e}_x} = \xhat, \quad
  \mathbf{\hat{e}_y} = \cos\omega\,\yhat + \sin\omega\,\zhat, \quad
  \mathbf{\hat{e}_z} = -\sin\omega\,\yhat + \cos\omega\,\zhat.
  \label{eq:smp_coord}
\end{equation}
From \Eq{eq:q_vector} and \Eq{eq:smp_coord}, we find the projection of 
$\mathbf{q}$ on the sample coordinates to be
\begin{align}
  q_x &= \mathbf{q}\cdot\mathbf{\hat{e}_x} 
       = q\cos\theta\cos\phi 
       \label{eq:qx} \\
  q_y &= \mathbf{q}\cdot\mathbf{\hat{e}_y} 
       = q\left(-\sin\theta\cos\omega + \cos\theta\sin\phi\sin\omega\right) 
       \label{eq:qy} \\
  q_z &= \mathbf{q}\cdot\mathbf{\hat{e}_z} 
       = q\left(\sin\theta\sin\omega + \cos\theta\sin\phi\cos\omega\right).
       \label{eq:qz}
\end{align}
With respect to the beam, the position on the detector is given by
\begin{equation}
  X = S \tan 2\theta \cos\phi, \quad Z = S \tan 2\theta \sin\phi.
\end{equation} 
The pixels on the detector are directly proportional to $X$ and $Z$. Thus,
these equations define the transformation rules from the detector space
to the sample q-space and vice versa.

For low angle x-ray scattering (LAXS), it is convenient to linearlize the above
equations in terms of $\theta$ and $\omega$. In the small angle approximation, 
$\sin\phi \approx Z/(2S\theta)$ and $\cos\phi \approx X/(2S\theta)$, and
\begin{align}
  q_x &\approx \frac{4\pi\theta\cos\phi}{\lambda} \approx kX/S \\
  q_y &\approx q_z\omega -\frac{4\pi\theta^2}{\lambda} \approx q_z\omega - \frac{\lambda q_z^2}{4\pi}\\
  q_z &\approx \frac{4\pi\theta\sin\phi}{\lambda} \approx kZ/S,
\end{align}
with $k=2\pi/\lambda$. 
In order to capture all Bragg peaks in one x-ray exposure, 
the sample was continuously rotated, which is equivalent to an integration over 
$\omega$. $\omega$ dependence appears through $q_y$, so rotating the 
sample amounts to an integration over $q_y$. $\omega$ ranges from 0 to
$2\theta$, where the substrate blocks the outgoing x-ray with its scattering
angle equal to $\theta$. Within the small angle approximation, 
$q \approx 4\pi\theta/\lambda \approx q_z$. Therefore, the integration limits 
for $q_y$ integration are $[-\lambda q_z^2/(4\pi), \lambda q_z^2/(4\pi)]$.
Integrating over $X$ and $Z$ is also necessary to accurately 
predict the correct intensity ratios. These lead to the observed intensity
written as,
\begin{align}
  I_{\mathrm{o},hk} 
    &\propto \int dX \int dZ \int d\omega |F_{hk}|^2 S_{hk}(\mathbf{q}) \nonumber \\
    &\propto |F_{hk}|^2 \int dq_x \int dq_z 
             \int_{-\frac{\lambda q_z^2}{4\pi}}^{\frac{\lambda q_z^2}{4\pi}} \frac{dq_y}{q_z} 
             S_{hk}(\mathbf{q}),
\end{align}
where $1/q_z$ factor in $q_y$ integration is the Lorentz polarization factor
in the small angle approximation. 

For a crystalline sample that possesses the azimuthal symmetry, the
structure factor is  
\begin{equation}
  S_{hk}(\mathbf{q}) = S_{hk}(q_r,q_z) 
  = \frac{1}{2\pi q_r}\delta(q_r-q_{r,k})\delta(q_z-q_{z,hk}),
\end{equation} 
where $q_{r,k}=2\pi |k|/\lambda_r$. Thus, the ripple scattering pattern is a 
collection of Bragg ``rings'' centered at the meridian, or $q_r=0$, and the 
Bragg peaks that are sometimes called the $k=0$ main peaks.  

The observed, integrated intensity of $hk$ peak is proportional to
\begin{equation}
  I_{\mathrm{o},hk} 
    \propto \frac{\lvert F_{hk} \rvert^2}{q_{z,hk}} \int\mathop{dq_x} 
            \int_{-q_{y0}}^{q_{y0}}
            \mathop{dq_y} \frac{\delta(q_r-q_{r,k})}{2\pi q_r},
\end{equation}
where $q_{y0} = \lambda q_{z,hk}^2/(4\pi)$.
For satellite peaks ($k \neq 0$), we have 
\begin{align}
  \int\mathop{dq_x} \int_{-q_{y0}}^{q_{y0}}\mathop{dq_y} \frac{\delta(q_r-q_{r,k})}{2\pi q_r}
  &\approx \int_{-\frac{q_{y0}}{q_{r,k}}}^{\frac{q_{y0}}{q_{r,k}}} \mathop{d\phi} 
          \int \mathop{dq_r} q_r\frac{\delta(q_r-q_{r,k})}{2\pi q_r} \\
 &= \frac{q_{y0}}{\pi q_{r,k}}.
\end{align}
For the main peaks ($k=0$), we have 
\begin{align}
  \int\mathop{dq_x} \int_{-q_{y0}}^{q_{y0}}\mathop{dq_y} \frac{\delta(q_r-q_{r,k})}{2\pi q_r}
  &= \int_0^{2\pi}\mathop{d\phi} \int\mathop{dq_r} q_r\frac{\delta(q_r-q_{r,k})}{2\pi q_r} \\
  &= 1
\end{align}
The observed intensity, $I_{\mathrm{o},hk}$, can be written as
\begin{align}
  I_{\mathrm{o},hk} &\propto \frac{|F_{hk}|^2}{q_{z,hk}}
                    \pars{\delta_{k0}+(1-\delta_{k0})\frac{\lambda q_{z,hk}^2}{4\pi^2 q_{r,k}}} \\
  &= |F_{hk}|^2 \pars{\frac{\delta_{k0}}{q_{z,hk}}+(1-\delta_{k0})\frac{\lambda q_{z,hk}}{2\pi}
  \frac{1}{2\pi q_{r,k}}}
\end{align}
The experimental observed form factor is written as 
\begin{align}
  |F_{h0}|^2 &\propto q_{z,h0} I_{\mathrm{o},h0} \label{eq:main}\\
  |F_{hk}|^2 &\propto \frac{2\pi q_{r,k}}{2\theta_{hk}} I_{\mathrm{o},hk} \label{eq:side} 
\end{align}
where $2\theta_{hk} = \lambda q_{z,hk}/(2\pi)$ is the incident angle at which 
the outgoing x-ray with its scattering angle equal to $\theta_{hk}$ gets 
blocked by the substrate. Eq.~(\ref{eq:main}) and (\ref{eq:side}) give the 
geometrical correction that must be applied to the observed intensity
in order to obtain the experimental form factor, which will be fitted to
a theoretical model of the electron density profile in the ripple phase. 

These factors are different from what Sengupta \textit{et al.} obtained. They
obtained $\Delta$ instead of $\omega_{\text{max}}$. This difference stems from
different substrates used in experiments. In this study, a thick Si wafer
was used, which does not permit x-ray to penetrate. This limits the range
of $\omega$, within which the outgoing wavevector actually gets to the
detector without being blocked by the substrate. In Sengupta \textit{et al.},
a glass beaker was used to orient their sample, which presumably is
transparent to x-rays at the energy used in their study. 

The unit cell vectors in the ripple phase can be expressed as 
\begin{equation}
  \mathbf{a} = \frac{D}{\tan\gamma}\xhat + D\zhat
\end{equation}
and
\begin{equation}
  \mathbf{b} = \lambda_r\xhat.
\end{equation}
The corresponding reciprocal lattice unit cell vectors are given by
\begin{equation}
  \mathbf{A} = \frac{2\pi}{D}\zhat
\end{equation}
and
\begin{equation}
  \mathbf{B} = \frac{2\pi}{\lambda_r}\xhat - \frac{2\pi}{\lambda_r\tan\gamma}\zhat.
\end{equation}
The reciprocal lattice vector, $\mathbf{q}_{hk}$ for the Bragg peak with 
Miller indices $(h,k)$ is 
\begin{equation}
  \mathbf{q}_{hk}=h\mathbf{A}+k\mathbf{B},
\end{equation}
so its components can be written as
\begin{equation}
  q_{x,k} = \frac{2\pi k}{\lambda_r}
\end{equation}
\begin{equation}
  q_y = 0
\end{equation}
\begin{equation}
  q_{z,hk} = \frac{2\pi h}{D} - \frac{2\pi k}{\lambda_r\tan\gamma}
\end{equation}

Figure shows a LAXS pattern from DMPC at 18 \degC. $D=59.2$ \AA. High 
resolution experiment. Up to $h=6$ orders were clearly observed in this
data set. 

Figure shows a LAXS pattern from DMPC at 18 \degC. $D=57.9$ \AA. Low
resolution experiment. Up to $h=9$ orders were clearly observed in this
data set. Because of a non-negligible degree of mosaicity in the sample,
strong orders cast their arcs over weaker orders. A care must be 
taken to decompose the intensity at a given pixel to intensity due to
a strong order's arc and to that due to a weak peak. This was achieved
by taking a $q_z$ swath and fitting the intensity to two Gaussian
functions whose widths were determined from the known instrumental 
resolution. Figure shows an example of this operation. Table and
Table show with and without the decomposition operation, respectively. 
For many of the orders observed, errors one would expect from 
neglecting the mosaicity effect were small. For higher orders, however,
this was crucial to obtain the correct integrated intensity.

In order to test the decomposition effect, fits were also performed
for the sum of intensity for orders that overlap. 

\section{Simple Delta Function Model}
Two models employed in this study were simple delta function (SDF) and
modified Gaussian function (MGF) models. As was shown previously,
both models results in the same phases.

\section{Simple 1G Model}
The 1G model approximates the lipid head 
and terminal methyl groups as a single 
Gaussian each. The methylene and water regions are both treated as a constant.
The gap between the two is represented by a sine function. The electron
density in the minus fluid convention is
\begin{equation}
  \rho(z) = \rhog(z) + \rhos(z),
\end{equation}
where
\begin{equation}
  \rhog(z) = \rhoh e^{-(z-\zh)^2/(2\sigmah^2)} + 
             \rhoh e^{-(z+\zh)^2/(2\sigmah^2)} + \rhom e^{-z^2/(2\sigmam^2)}
\end{equation}
and
\begin{equation}
  \rhos(z) =  \left\{
    \begin{array}{ccc}
      0 & \text{if } & -D/2 \leq z < -\zw,\\
      -\Delta\rho\braces{\sin\bracks{\pi(z+\zh)/w}+1}/2 
        & \text{if } & -\zw \leq z < -\zchtwo,\\
      -\Delta\rho & \text{if } & -\zchtwo \leq z < \zchtwo,\\
      \Delta\rho\braces{\sin\bracks{\pi(z-\zh)/w}-1}/2 
        & \text{if } & \zchtwo \leq z < \zw,\\
      0 & \text{if } & \zw \leq z \leq D/2,
    \end{array}
  \right.
\end{equation}
with $\zw=\zh+w/2$, $\zchtwo=\zh-w/2$, and $\Delta\rho=\rhow-\rhochtwo$. 
Note that in the minus fluid convenction the electron density 
of lipid bilayers are measured relative to that of water. Also note that 
$\rhos(z)$ and its first derivative are continuous.

The transbilayer profile along $x=-z\tan\psi$ can be obtained by rotating
the coordinates $x$ and $z$ by $\psi$ in the clockwise direction and
reexpressing $\rho(z)$ in terms of the rotated coordinates. This leads
to replaincg $x$ with $x'=x\cos\psi+z\sin\psi$ and
$z$ with $z'=-x\sin\psi+z\cos\psi$. The rotated transbilayer profile is
\begin{equation}
  \rho(x,z) = \delta(x+z\tan\psi)\left[\rhog(x,z) + \rhos(x,z)\right],
\end{equation}
where the Gaussian part of the model is 
\begin{equation}
  \rhog(x,z) = \rhoh e^{-(z'-\zh)^2/(2\sigmah^2)} \\
  + \rhoh e^{-(z'+\zh)^2/(2\sigmah^2)}
  - \rhom e^{-z'^2/(2\sigmam^2)},
\end{equation}
and the strip part
\begin{equation}
  \rhos(x,z) =  \left\{
    \begin{array}{ccc}
      0 & \text{if } & -D/2 \leq z' < -\zw,\\
      -\Delta\rho\braces{\sin\bracks{\pi(z'+\zh)/w}+1}/2 
        & \text{if } & -\zw \leq z' < -\zchtwo,\\
      -\Delta\rho & \text{if } & -\zchtwo \leq z' < \zchtwo,\\
      \Delta\rho\braces{\sin\bracks{\pi(z'-\zh)/w}-1}/2 
        & \text{if } & \zchtwo \leq z' < \zw,\\
      0 & \text{if } & \zw \leq z' \leq D/2.
    \end{array}
  \right.
\end{equation}
The transbilayer part of the form factor is
\begin{align}
  F_\mathrm{T} 
  &= \int_{-\frac{D}{2}}^{\frac{D}{2}} \int_{-\frac{\lambda_r}{2}}^{\frac{\lambda_r}{2}} 
     \rho(x,z) e^{i(q_xx+q_zz)} dxdz \nonumber\\
  &= 2 \int_0^{\frac{D}{2}} \rho(z) \cos (q_zz) dz
\end{align}
The second eqality follows from the inversion symmetry of $\rho(z)$. 
\begin{align}
  \frac{F_\mathrm{s}}{2} 
  &= \int_0^{\frac{D}{2}} \rho_\mathrm{s}(z) \cos (q_zz) dz \nonumber\\
  &= -\Delta\rho \int_0^{\zchtwo} \cos(q_zz) dz 
     + \frac{\Delta\rho}{2} \int_{\zchtwo}^{\zw} 
       \braces{\sin\bracks{\frac{\pi}{w}(z-\zh)}-1} \cos(q_zz) dz \nonumber\\
  &=
\end{align}

\begin{multline}
  \int \sin\bracks{\frac{\pi}{w}(z-\zh)} \cos q_zz dz
  = \left(1-\frac{\pi^2}{w^2q_z^2}\right)^{-1} 
  \left\{\frac{1}{q} \sin\bracks{\frac{\pi}{w}(z-\zh)} \sin q_zz\right. \\
  + \left.\frac{\pi}{wq^2}\cos\bracks{\frac{\pi}{w}(z-\zh)}\cos q_zz\right\}
\end{multline}
Non-rotated $F_s$ is
\begin{equation}
  F_\mathrm{s} = 2\Delta\rho\sin(q_z\zh)\cos\pars{\frac{q_zw}{2}}
                 \left[-\frac{1}{q}+\frac{qw^2}{q^2w^2-\pi^2}\right] 
\end{equation}
The rotated $F_G$ is
\begin{equation}
  F_\mathrm{G} = \sqrt{2\pi}\rhom\cos\psi \pars{
  \rhm\sigmah\cos(\zh\theta) e^{-\frac{1}{2}\sigmah^2\theta^2}
  -\sigmam e^{-\frac{1}{2}\sigmam^2\theta^2}}
\end{equation}
with $\theta(\mathbf{q})=\cos\psi(q_z-q_x\tan\psi)$.

\section{fit}

\appendix
\section{Rotation of a Two-Dimensional Function}
Let us consider rotating a function, $f(x,z)$ in two dimensions by an angle, 
$\psi$, in the counterclockwise direction (see Fig. X). This is easily 
achieved by rotating the coordinate system by $\psi$ in the clockwise direction. 
Let rotated coordinates be $x'$ and $z'$. A point in the original coodinates,
($x$, $z$), is written as ($x'$, $z'$) in the new coordinates. More specifically,
the point P is written as 
$\mathbf{P}=x\xhat+z\zhat=x'\xhat'+z'\zhat'$. $\xhat$ and $\zhat$ in
the $x'z'$ coordinate system are written as 
\begin{align}
  \xhat &= \cos\psi\xhat'+\sin\psi\zhat' \\
  \zhat &= -\sin\psi\xhat'+\cos\psi\zhat'.
\end{align}
Pluggin these in $\mathbf{P}=x\xhat+z\zhat$ leads to
\begin{align}
  x' &= x\cos\psi - z\sin\psi \\
  z' &= z\cos\psi + x\sin\psi,
\end{align}
the inverse of which is
\begin{align}
  x &= x'\cos\psi + z'\sin\psi \\
  z &= -x'\sin\psi + z'\cos\psi.
\end{align}
Using the latter equations, $f(x,z)$ can be expressed in terms of $x'$ and $z'$. 
The resulting function $f(x',z')$ is the rotated version of $f(x,z)$. 

As an 
example, let us consider a Dirac delta function located at $(x,z)=(0,\zh)$,
that is, $f(x,z)=\delta(x)\delta(z-\zh)$. After the rotation by $\psi$, it 
becomes
\begin{align*}
  f(x,z) 
  &\rightarrow 
    \delta(x\cos\psi+z\sin\psi) \delta(-x\sin\psi+z\cos\psi-\zh) \\
  &= \frac{\delta(x+z\tan\psi)}{|\cos\psi|}
     \frac{\delta(-x\sin\psi\cos\psi+z\cos^2\psi-\zh\cos\psi)}{1/|\cos\psi|} \\
  &= \delta(x+z\tan\psi)\delta(z\tan\psi\sin\psi\cos\psi+z\cos^2\psi-\zh\cos\psi) \\
  &= \delta(x+z\tan\psi)\delta(z-\zh\cos\psi),
\end{align*}
which is a part of the expression for $T_\psi(x,z)$ in the simple delta 
function model.
\end{document}

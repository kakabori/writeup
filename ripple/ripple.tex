\documentclass[12pt,letterpaper]{article}
%%%%%%%%%%%%%%%%% common packages are included here %%%%%%%%%%%%%%%%%
\usepackage{graphicx}
\usepackage{epstopdf}
\usepackage{tabularx}
\usepackage[lofdepth,lotdepth]{subfig} % subfloat
\usepackage{bm} % bold math
\usepackage{color}
\usepackage[centertags]{amsmath}
\usepackage{mathrsfs}
\usepackage{amsmath}
\usepackage{mathtools}
\usepackage{amssymb}
\usepackage{amsfonts}
\usepackage{amssymb}
\usepackage{amsthm}
\usepackage{newlfont}
\usepackage{textcomp,gensymb} % for \textcelsius, \textdegree, and \degree
\usepackage{syntonly}

%%%%%%%%%%%%%%%%%%%% some declaration %%%%%%%%%%%%%%%%%%%%%
\DeclareGraphicsExtensions{.pdf,.png,.jpg,.eps}
\graphicspath{{./figures/}}

%%%%%%%%%%%%%%%%%%% new commands are defined here %%%%%%%%%%%%%%%%%%%%%
\newcommand{\dg}{$^{\circ}$} % degree symbol
\newcommand{\iang}{\AA$^{-1}$} % inverse Angstrom symbol
\newcommand{\degC}{$^{\circ}\mathrm{C}$} % degree Celcius
\newcommand{\Eq}[1]{Eq.\,(\ref{#1})} % reference to an equation

% Some mathematical (physical) quantities and symbols that are used often
\newcommand{\xhat}{\mathbf{\hat{x}}}
\newcommand{\yhat}{\mathbf{\hat{y}}}
\newcommand{\zhat}{\mathbf{\hat{z}}}
\newcommand{\kin}{\mathbf{k}_{\mathrm{in}}}
\newcommand{\kout}{\mathbf{k}_{\mathrm{out}}}
\newcommand{\Tat}{\mathrm{Tat}}
\newcommand{\DOPC}{\mathrm{DOPC}}
\newcommand{\cm}{\mathrm{cm}}

% To simplify some formatting issues
\newcommand{\pars}[1]{\mathopen{}\left( #1 \right)\mathclose{}} % () without extra spaces due to \left and \right
\newcommand{\angles}[1]{\left\lange #1 \right\rangle}%      <>
\newcommand{\braces}[1]{\left\lbrace #1 \right\rbrace}%     {}
\newcommand{\bracks}[1]{\left\lbrack #1 \right\rbrack}%     []
\newcommand{\ds}[1]{\displaystyle{#1}}%
\newcommand{\+}{^{\dagger}}%                                
\newcommand{\partiald}[3][]{{\partial^{#1}#2 \over \partial {#3}^{#1}}}%

%\syntaxonly
\usepackage[pdftex]{hyperref}

\begin{document}
\today

\section{setup}
The incoming and outgoing wavevectors of the x-ray beam in Fig. XXX 
are given by
\begin{equation}
  \kin = \frac{2\pi}{\lambda} \yhat, \quad
  \kout = 
    \frac{2\pi}{\lambda} \left( 
      \sin 2\theta \cos\phi \, \xhat
      + \cos 2\theta \, \yhat
      + \sin 2\theta \sin\phi \, \zhat 
    \right),
  \label{eq:kinkout}
\end{equation}
where $\lambda$ is the wavelength of x-ray. The scattering vector is
the difference between $\kin$ and $\kout$,
\begin{align}
  \mathbf{q} &= \kout - \kin \nonumber \\
             &= q \left( 
                  \cos\theta\cos\phi \, \xhat - \sin\theta \, \yhat
                  + \cos\theta\sin\phi \, \zhat
                \right),
  \label{eq:q_vector}
\end{align}
where $q=4\pi\sin\theta/\lambda$ is the magnitude of the scattering vector. 
When the sample is rotated by $\omega$ about the x-axis in the clockwise 
direction as shown in Fig. XXX, the sample coodinates written in terms of 
the lab coordinates are  
\begin{equation}
  \mathbf{\hat{e}_x} = \xhat, \quad
  \mathbf{\hat{e}_y} = \cos\omega\,\yhat + \sin\omega\,\zhat, \quad
  \mathbf{\hat{e}_z} = -\sin\omega\,\yhat + \cos\omega\,\zhat.
  \label{eq:smp_coord}
\end{equation}
From \Eq{eq:q_vector} and \Eq{eq:smp_coord}, we find the projection of 
$\mathbf{q}$ on the sample coordinates to be
\begin{align}
  q_x &= \mathbf{q}\cdot\mathbf{\hat{e}_x} 
       = q\cos\theta\cos\phi 
       \label{eq:qx} \\
  q_y &= \mathbf{q}\cdot\mathbf{\hat{e}_y} 
       = q\left(-\sin\theta\cos\omega + \cos\theta\sin\phi\sin\omega\right) 
       \label{eq:qy} \\
  q_z &= \mathbf{q}\cdot\mathbf{\hat{e}_z} 
       = q\left(\sin\theta\sin\omega + \cos\theta\sin\phi\cos\omega\right).
       \label{eq:qz}
\end{align}
With respect to the beam, the position on the detector is given by
\begin{equation}
  X = S \tan 2\theta \cos\phi, \quad Z = S \tan 2\theta \sin\phi.
\end{equation} 
The pixels on the detector are directly proportional to $X$ and $Z$. Thus,
these equations define the transformation rules from the detector space
to the sample q-space and vice versa.

For low angle x-ray scattering (LAXS), it is convenient to linearlize the above
equations in terms of $\theta$ and $\omega$. In the small angle approximation, 
we have $q_x \approx kx$, $q_y \approx q_z\omega$, and $q_z \approx kz$, where
$k=2\pi/\lambda$, $x=X/S$, and $z=Z/S$. The observed intensity is equal to
the integration of intensity at a given angle over $X$, $Z$, and $\omega$, 
that is, 
\begin{align}
  I_o(h,k) &\sim \int dX \int dZ \int d\omega |F(h,k)|^2 S(h,k) \nonumber \\
           &\sim \int dq_x \int dq_z \int \frac{dq_y}{q_z} \sum_{h,k} |F_{hk}|^2 S_{hk}(\mathbf{q}),
\end{align}
where $1/q_z$ factor in $q_y$ integration is the Lorentz polarization factor
in the small angle approximation. 

The unit cell vectors in the ripple phase can be expressed as 
\begin{equation}
  \mathbf{a} = \frac{D}{\tan\gamma}\xhat + D\zhat
\end{equation}
and
\begin{equation}
  \mathbf{b} = \lambda_r\xhat.
\end{equation}
The corresponding reciprocal lattice unit cell vectors are given by
\begin{equation}
  \mathbf{A} = \frac{2\pi}{D}\zhat
\end{equation}
and
\begin{equation}
  \mathbf{B} = \frac{2\pi}{\lambda_r}\xhat - \frac{2\pi}{\lambda_r\tan\gamma}\zhat.
\end{equation}
The reciprocal lattice vector, $\mathbf{q}_{hk}$ for the Bragg peak with 
Miller indices $(h,k)$ is 
\begin{equation}
  \mathbf{q}_{hk}=h\mathbf{A}+k\mathbf{B},
\end{equation}
so its components can be written as
\begin{equation}
  q_{x,k} = \frac{2\pi k}{\lambda_r}
\end{equation}
\begin{equation}
  q_y = 0
\end{equation}
\begin{equation}
  q_{z,hk} = \frac{2\pi h}{D} - \frac{2\pi k}{\lambda_r\tan\gamma}
\end{equation}

Since the sample is crystalline in the ripple phase, the x-ray structure factor
from a domain whose Cartesian coordinates coincide with the lab Cartesian 
coordinates is 
\begin{equation}
  S(q_x,q_y,q_z) = \sum_{h,k} \delta(q_x-q_{x,k}) \delta(q_y) \delta(q_z-q_{z,hk}),
\end{equation}
where the ripple direction is assumed to be along x-axis. Since the sample 
possesses the azimuthal symmetry (in other words, in-plane powder), the
structure factor becomes
\begin{equation}
  S(q_r,q_z) = \sum_{h,k} \frac{1}{2\pi q_r}\delta(q_r-q_{r,k})\delta(q_z-q_{z,hk}),
\end{equation} 
where $q_{r,k}=2\pi |k|/\lambda_r$. 

The observed intensity is proportional to
\begin{equation}
  I_{\mathrm{o}} \sim \sum_{h,k} \frac{\lvert F_{hk} \rvert^2}{q_{z,hk}} \int\mathop{dq_x} 
                      \int_{-a}^{a}\mathop{dq_y} \frac{\delta(q_r-q_{r,k})}{2\pi q_r}
\end{equation}
For satellite peaks ($k \neq 0$), we have 
\begin{align}
  \int\mathop{dq_x} \int_{-a}^a\mathop{dq_y} \frac{\delta(q_r-q_{r,k})}{2\pi q_r}
  &= \int_{-\Delta\phi}^{\Delta\phi}\mathop{d\phi} \int\mathop{dq_r} q_r\frac{\delta(q_r-q_{r,k})}{2\pi q_r} \\
  &= \frac{\Delta\phi}{\pi} \\
  &\approx \frac{a}{\pi q_{r,k}}
\end{align}
where $a=\lambda q_{z,hk}^2/(4\pi)$. Note that the small angle approximation was used here for $\Delta\phi$.
For the main peaks ($k=0$), we have 
\begin{align}
  \int\mathop{dq_x} \int_{-a}^a\mathop{dq_y} \frac{\delta(q_r-q_{r,k})}{2\pi q_r}
  &= \int_0^{2\pi}\mathop{d\phi} \int\mathop{dq_r} q_r\frac{\delta(q_r-q_{r,k})}{2\pi q_r} \\
  &= 1
\end{align}
The observed intensity, $I_{\mathrm{o},hk}$, can be written as
\begin{align}
  I_{\mathrm{o},hk} &\sim \sum_{h,k} \frac{|F_{hk}|^2}{q_{z,hk}}
                    \pars{\delta_{k0}+(1-\delta_{k0})\frac{\lambda q_{z,hk}^2}{4\pi^2 q_{r,k}}} \\
  &= \sum_{h,k}|F_{hk}|^2 \pars{\frac{\delta_{k0}}{q_{z,hk}}+(1-\delta_{k0})\frac{\lambda q_{z,hk}}{2\pi}
  \frac{1}{2\pi q_{r,k}}}
\end{align}
Then, the form factor is 
\begin{align}
  |F_{h0}|^2 &\sim q_{z,h0} I_{\mathrm{o},h0} \\
  |F_{hk}|^2 &\sim \frac{2\pi q_{r,k} I_{\mathrm{o},hk}}{\omega_{\mathrm{max},hk}} 
\end{align}

These factors are different from what Sengupta \textit{et al.} obtained. They
obtained $\Delta$ instead of $\omega_{\text{max}}$. This difference stems from
different substrates used in experiments. In this study, a thick Si wafer
was used, which does not permit x-ray to penetrate. This limits the range
of $\omega$, within which the outgoing wavevector actually gets to the
detector without being blocked by the substrate. In Sengupta \textit{et al.},
a glass beaker was used to orient their sample, which presumably is
transparent to x-rays at the energy used in their study. 

Figure shows a LAXS pattern from DMPC at 18 \degC. $D=59.2$ \AA. High 
resolution experiment. Up to $h=6$ orders were clearly observed in this
data set. 

Figure shows a LAXS pattern from DMPC at 18 \degC. $D=57.9$ \AA. Low
resolution experiment. Up to $h=9$ orders were clearly observed in this
data set. Because of a non-negligible degree of mosaicity in the sample,
strong orders cast their arcs over weaker orders. A care must be 
taken to decompose the intensity at a given pixel to intensity due to
a strong order's arc and to that due to a weak peak. This was achieved
by taking a $q_z$ swath and fitting the intensity to two Gaussian
functions whose widths were determined from the known instrumental 
resolution. Figure shows an example of this operation. Table and
Table show with and without the decomposition operation, respectively. 
For many of the orders observed, errors one would expect from 
neglecting the mosaicity effect were small. For higher orders, however,
this was crucial to obtain the correct integrated intensity.

In order to test the decomposition effect, fits were also performed
for the sum of intensity for orders that overlap. 

Two models employed in this study were simple delta function (SDF) and
modified Gaussian function (MGF) models. As was shown previously,
both models results in the same phases.

\end{document}
\documentclass[12pt,letterpaper]{article}
\author{Kiyo Akabori}
\title{Mosaic Spread Analysis}
\date{\today}
%-------------------------------------Package-----------------------
\usepackage{graphicx}
\usepackage{epstopdf}
\usepackage{tabularx}
\usepackage[lofdepth,lotdepth]{subfig}%subfloat
\usepackage{bm}%bold math
\usepackage{color}
\usepackage[centertags]{amsmath}
\usepackage{mathrsfs}
\usepackage{amsmath}
\usepackage{mathtools}
\usepackage{amssymb}
\usepackage{amsfonts}
\usepackage{amssymb}
\usepackage{amsthm}
\usepackage{newlfont}
\usepackage{textcomp,gensymb}%for \textdegree and \degree

\DeclareGraphicsExtensions{.pdf,.png,.jpg,.eps}
\graphicspath{{./figures/}}

%%%%%%%%%%%%%%%%%%%%%%%% new command %%%%%%%%%%%%%%%%%%%%%%%%%%%%%%%%%%%%%%%%%%         
%\newcommand{\dg}{^{\circ}}%degree symbol
\newcommand{\degC}{$^{\circ}\mathrm{C}$}
\newcommand{\iang}{\textrm{\AA}^{-1}} % inverse Angstrom symbol
\newcommand{\Eq}[1]{Eq.\,(\ref{#1})} % reference to an equation
\newcommand{\xhat}{\mathbf{\hat{x}}}
\newcommand{\yhat}{\mathbf{\hat{y}}}
\newcommand{\zhat}{\mathbf{\hat{z}}}
\newcommand{\kin}{\mathbf{k}_{\mathrm{in}}}
\newcommand{\kout}{\mathbf{k}_{\mathrm{out}}}

\newcommand{\pars}[1]{\mathopen{}\left( #1 \right)\mathclose{}} % () without extra spaces due to \left and \right

\newcommand{\angles}[1]{\left\lange #1 \right\rangle}%      <>
\newcommand{\braces}[1]{\left\lbrace #1 \right\rbrace}%     {}
\newcommand{\bracks}[1]{\left\lbrack #1 \right\rbrack}%     []
\newcommand{\ds}[1]{\displaystyle{#1}}%
\newcommand{\+}{^{\dagger}}%                                
\newcommand{\partiald}[3][]{{\partial^{#1}#2 \over \partial {#3}^{#1}}}%

%%%%%%%%%%%%%%%%%%%%%%%%%%%%%%%%%%%%%%%%%%%%%%%%%%%%%%%%%%%%%%%%%%%%%%%%%%%%%%%
\begin{document}
%\maketitle

\section{setup}
The incoming and outgoing wavevectors of the x-ray beam in Fig. XXX 
are given by
\begin{equation}
  \kin = \frac{2\pi}{\lambda} \yhat, \quad
  \kout = 
    \frac{2\pi}{\lambda} \left( 
      \sin 2\theta \cos\phi \, \xhat
      + \cos 2\theta \, \yhat
      + \sin 2\theta \sin\phi \, \zhat 
    \right),
  \label{eq:kinkout}
\end{equation}
where $\lambda$ is the wavelength of x-ray. The scattering vector is
the difference between $\kin$ and $\kout$,
\begin{align}
  \mathbf{q} &= \kout - \kin \nonumber \\
             &= q \left( 
                  \cos\theta\cos\phi \, \xhat - \sin\theta \, \yhat
                  + \cos\theta\sin\phi \, \zhat
                \right),
  \label{eq:q_vector}
\end{align}
where $q=4\pi\sin\theta/\lambda$ is the magnitude of the scattering vector. 
When the sample is rotated by $\omega$ about the x-axis in the clockwise 
direction as shown in Fig. XXX, the sample coodinates written in terms of 
the lab coordinates are  
\begin{equation}
  \mathbf{\hat{e}_x} = \xhat, \quad
  \mathbf{\hat{e}_y} = \cos\omega\,\yhat + \sin\omega\,\zhat, \quad
  \mathbf{\hat{e}_z} = -\sin\omega\,\yhat + \cos\omega\,\zhat.
  \label{eq:smp_coord}
\end{equation}
From \Eq{eq:q_vector} and \Eq{eq:smp_coord}, we find the projection of 
$\mathbf{q}$ on the sample coordinates to be
\begin{align}
  q_x &= \mathbf{q}\cdot\mathbf{\hat{e}_x} 
       = q\cos\theta\cos\phi 
       \label{eq:qx} \\
  q_y &= \mathbf{q}\cdot\mathbf{\hat{e}_y} 
       = q\left(-\sin\theta\cos\omega + \cos\theta\sin\phi\sin\omega\right) 
       \label{eq:qy} \\
  q_z &= \mathbf{q}\cdot\mathbf{\hat{e}_z} 
       = q\left(\sin\theta\sin\omega + \cos\theta\sin\phi\cos\omega\right).
       \label{eq:qz}
\end{align}
With respect to the beam, the position on the detector is given by
\begin{equation}
  X = S \tan 2\theta \cos\phi, \quad Z = S \tan 2\theta \sin\phi.
\end{equation} 
The pixels on the detector are directly proportional to $X$ and $Z$. Thus,
these equations define the transformation rules from the detector space
to the sample q-space and vice versa.

For low angle x-ray scattering (LAXS), it is convenient to linearlize the above
equations in terms of $\theta$ and $\omega$. In the small angle approximation, 
we have $q_x \approx kx$, $q_y \approx q_z\omega$, and $q_z \approx kz$, where
$k=2\pi/\lambda$, $x=X/S$, and $z=Z/S$. The observed intensity is equal to
the integration of intensity at a given angle over $X$, $Z$, and $\omega$, 
that is, 
\begin{align}
  I_o(h,k) &\sim \int dX \int dZ \int d\omega |F(h,k)|^2 S(h,k) \nonumber \\
           &\sim \int dq_x \int dq_z \int \frac{dq_y}{q_z} |F(h,k)|^2 S(h,k),
\end{align}
where $1/q_z$ factor in $q_y$ integration is the Lorentz polarization factor
in the small angle approximation. This shows that the Lorentz polarization 
factor stems from the sample rotation. An intuitive way to understand the
origin of this factor is illustrated in Fig. , where the Ewald sphere 
is employed. A given Bragg peak is smeared in the reciprocal space due to 
the sample rotation, and the extent of this smearing is larger for 
higher $q_z$. The intensity for a peak is observed when the smeared peak
crosses the Ewald sphere. Since the total number of scattering bodies 
is the same for different orders, the apparent intensity is weaker for
higher $q_z$ orders, which leads to $1/q^z$ factor.

Scattering from a domain whose Cartesian coordinates coincide with the lab Cartesian 
coordinates is given by
\begin{equation}
  S(q_x,q_y,q_z) = \pars{\delta(q_x)\delta\pars{q_z-\frac{2\pi h}{D}}
                   + \delta\pars{q_x-}\delta\pars{q_z}}\delta(q_y),
\end{equation}
where the ripple direction is assumed to be along x-axis. Since the sample 
possesses the azimuthal symmetry, 
\begin{equation}
  S(q_r,q_z) = \delta\pars{q_z-\frac{2\pi h}{D}}\delta(q_r)
               + \frac{1}{2\pi q_r}\delta\pars{q_z...}\delta\pars{q_r-q_0}
\end{equation} 
(fill in $q_o$ later. There might also be some $q_r$ factor somewhere). 
The limits of $q_y$ integration is important. It goes
from 0 to $q_z\omega_0$, where $\omega_0$ is the maximum angle of incidence. 
$q_x$ and $q_y$ integrations lead to $\Delta\phi q_0$, but $\Delta\phi = q_z\omega_0/q_0$,
resulting in $\omega_0$. At the end, we get
\begin{equation}
  I_o \sim \frac{1}{q_z(h)}|F(h,0)|^2+\frac{\omega_0}{2\pi q_r(k)}|F(h,k\neq 0)|^2
\end{equation}
This is indeed what Sengupta \textit{et al.} obtained [citation]. 

Figure shows a LAXS pattern from DMPC at 18 \degC. $D=59.2$ \AA. High 
resolution experiment. Up to $h=6$ orders were clearly observed in this
data set. 

Figure shows a LAXS pattern from DMPC at 18 \degC. $D=57.9$ \AA. Low
resolution experiment. Up to $h=9$ orders were clearly observed in this
data set. Because of a non-negligible degree of mosaicity in the sample,
strong orders cast their arcs over weaker orders. A care must be 
taken to decompose the intensity at a given pixel to intensity due to
a strong order's arc and to that due to a weak peak. This was achieved
by taking a $q_z$ swath and fitting the intensity to two Gaussian
functions whose widths were determined from the known instrumental 
resolution. Figure shows an example of this operation. Table and
Table show with and without the decomposition operation, respectively. 
For many of the orders observed, errors one would expect from 
neglecting the mosaicity effect were small. For higher orders, however,
this was crucial to obtain the correct integrated intensity.

In order to test the decomposition effect, fits were also performed
for the sum of intensity for orders that overlap. 

Two models employed in this study were simple delta function (SDF) and
modified Gaussian function (MGF) models. As was shown previously,
both models results in the same phases.

\end{document}